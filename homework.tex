%%% Template adapted from https://github.com/jdavis/latex-homework-template

%%% HOW TO USE THIS TEMPLATE

%%% Fill in your name, UIN, and the homework number in the below section

\newcommand{\hmwkTitle}{\textbf{Homework 1}}
\newcommand{\hmwkClass}{Discrete Structures}
\newcommand{\hmwkClassInstructor}{Kumar}
\newcommand{\hmwkAuthorName}{\textbf{Student Student}}
\newcommand{\hmwkAuthorUIN}{\textbf{UIN}}

%%% In between the \begin{document} and \end{document} tags,
%%% use the following template for each problem. The number in brackets
%%% should be the problem number. 

%%% \begin{homeworkProblem}[1]
%%%    \textbf{Solution:}
%%% \end{homeworkProblem}

%%% If you have any problems with this template,
%%% email me at larenspear@tamu.edu or contact me on Slack.

\documentclass{article}

\usepackage{fancyhdr}
\usepackage{extramarks}
\usepackage{amsmath}
\usepackage{amsthm}
\usepackage{amsfonts}
\usepackage[plain]{algorithm}
\usepackage{algpseudocode}

%
% Basic Document Settings
%

\topmargin=-0.45in
\evensidemargin=0in
\oddsidemargin=0in
\textwidth=6.5in
\textheight=9.0in
\headsep=0.25in

\linespread{1.1}

\pagestyle{fancy}
\lhead{\hmwkAuthorName \ (\hmwkAuthorUIN)}
\chead{\hmwkTitle}
\rhead{\hmwkClass\ (\hmwkClassInstructor)}
\lfoot{\lastxmark}
\cfoot{\thepage}

\renewcommand\headrulewidth{0.4pt}
\renewcommand\footrulewidth{0.4pt}

\setlength\parindent{0pt}

%
% Create Problem Sections
%

\newcommand{\enterProblemHeader}[1]{
    \nobreak\extramarks{}{Problem \arabic{#1} continued on next page\ldots}\nobreak{}
    \nobreak\extramarks{Problem \arabic{#1} (continued)}{Problem \arabic{#1} continued on next page\ldots}\nobreak{}
}

\newcommand{\exitProblemHeader}[1]{
    \nobreak\extramarks{Problem \arabic{#1} (continued)}{Problem \arabic{#1} continued on next page\ldots}\nobreak{}
    \stepcounter{#1}
    \nobreak\extramarks{Problem \arabic{#1}}{}\nobreak{}
}

\setcounter{secnumdepth}{0}
\newcounter{partCounter}
\newcounter{homeworkProblemCounter}
\setcounter{homeworkProblemCounter}{1}

% Homework Problem Environment

\newenvironment{homeworkProblem}[1][-1]{
    \ifnum#1>0
        \setcounter{homeworkProblemCounter}{#1}
    \fi
    \section{Problem \arabic{homeworkProblemCounter}}
    \setcounter{partCounter}{1}
}{
}

\renewcommand{\part}[1]{\textbf{\large Part \Alph{partCounter}}\stepcounter{partCounter}\\}

%
% Various Helper Commands
%

% Useful for algorithms
\newcommand{\alg}[1]{\textsc{\bfseries \footnotesize #1}}

% For derivatives
\newcommand{\deriv}[1]{\frac{\mathrm{d}}{\mathrm{d}x} (#1)}

% For partial derivatives
\newcommand{\pderiv}[2]{\frac{\partial}{\partial #1} (#2)}

% Integral dx
\newcommand{\dx}{\mathrm{d}x}

% Alias for the Solution section header
\newcommand{\solution}{\textbf{\large Solution}}

% Probability commands: Expectation, Variance, Covariance, Bias
\newcommand{\E}{\mathrm{E}}
\newcommand{\Var}{\mathrm{Var}}
\newcommand{\Cov}{\mathrm{Cov}}
\newcommand{\Bias}{\mathrm{Bias}}

\begin{document}

\begin{homeworkProblem}[1]

    \textbf{Solution:}
    
\end{homeworkProblem}

\begin{homeworkProblem}[2]

    \textbf{Solution:}
    
\end{homeworkProblem}

\begin{homeworkProblem}[3]

    \textbf{Solution:}
    
\end{homeworkProblem}

\begin{homeworkProblem}[4]

    \textbf{Solution:}
    
\end{homeworkProblem}

\begin{homeworkProblem}[5]

    \textbf{Solution:}
    
\end{homeworkProblem}

\begin{homeworkProblem}[6]

    \textbf{Solution:}
    
\end{homeworkProblem}

\begin{homeworkProblem}[7]

    \textbf{Solution:}
    
\end{homeworkProblem}

\begin{homeworkProblem}[8]

    \textbf{Solution:}
    
\end{homeworkProblem}

\end{document}
